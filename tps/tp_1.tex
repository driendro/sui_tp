\chapter{Trabajo Practico 1}

\textbf{Conjuntos numericos, propiedades, operaciones combinadas, MCM, MCD}

\section{Ejercicio 1}

Colocar $\in$, $\notin$, $\subset$, $\not\subset$ según corresponda


\begin{tabular}{rccr|cccrccr}
	\centering
	a) & $\mathbb N$ & ... & $ \mathbb Z  $ &  &  &  & e) & $      1.3                            $ & ... & $ \mathbb Q  $ \\
	b) & $       -3$ & ... & $ \mathbb N_0$ &  &  &  & f) & $\sqrt{2}                             $ & ... & $ \mathbb Q  $ \\
	c) & $\mathbb Q$ & ... & $ \mathbb Z  $ &  &  &  & g) & $4.5; \; \frac{2}{3}; \; 2.\widehat{3}$ & ... & $ \mathbb Z  $ \\
	d) & $\mathbb I$ & ... & $ \mathbb R  $ &  &  &  & h) & $\mathbb R                            $ & ... & $ \mathbb Q  $ \\
\end{tabular}

\begin{equation}
	\begin{split}
		M=a+b &= 3+2 \\
		      &= 5 \\
	\end{split}
\end{equation}


%
%\begin{enumerate}
%    \item[a] $\mathbb N                             \; \; \; ... \; \; \; \mathbb Z  $
%    \item[b] $       -3                             \; \; \; ... \; \; \; \mathbb N_0$
%    \item[c] $\mathbb Q                             \; \; \; ... \; \; \; \mathbb Z  $
%    \item[d] $\mathbb I                             \; \; \; ... \; \; \; \mathbb R  $
%    \item[e] $      1.3                             \; \; \; ... \; \; \; \mathbb Q  $
%    \item[f] $\sqrt{2}                              \; \; \; ... \; \; \; \mathbb Q  $
%    \item[g] $4.5; \; \frac{2}{3}; \; 2.\widehat{3} \; \; \; ... \; \; \; \mathbb Z  $
%    \item[h] $\mathbb R                             \; \; \; ... \; \; \; \mathbb Q  $
%\end{enumerate}
%
%\section{Ejercicio n°2}
%\textsl{Convertir en fracción irreducible las siguientes expresiones decimales:}
%\section{Ejercicio n°3}
%\textsl{Hallar el valor de las siguientes potencias}
%\section{Ejercicio n°4}
%\textsl{Ejercicios combinados:}
%\section{Ejercicio n°5}
%\textsl{Resolver, aplicando convenientemente propiedades:}
%\section{Ejercicio n°6}
%\textsl{Calcular}
%\section{Ejercicio n°7}
%\textsl{Determinar el resultado de las siguientes expresiones}
%\section{Ejercicio n°8}
%\textsl{Plantear y resolver}
%\section{Ejercicio n°9}
%\textsl{Hallarel máximo común divisor (MCD) y el mínimo común múltiplo de los siguientes valores:}
%\section{Ejercicio n°10}
%\textsl{Analizar las siguientes situaciones problemáticas}
